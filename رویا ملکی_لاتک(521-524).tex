\documentclass[11pt]{article}
\usepackage{multicol}
\usepackage{xcolor}
\usepackage{graphicx}

\usepackage{amsmath}
\usepackage{color}


\begin{document}


\begin{flushright}
 \texttt{Computability and Undecidability} \hspace*{0.1cm}\textbf{$|$} \hspace*{0.1cm} \textbf{521}\hspace*{0.1cm}
\end{flushright}
\vspace*{0.5cm}

\begin{multicols}{2}

\hspace*{0.5cm}
b) Complement of a recursive language is
recursive\\

\vspace*{0.1cm}
\hspace*{0.5cm}
c) Recursive language may loop forever
on Turing machine\\

\vspace*{0.1cm}
\hspace*{0.5cm}
d) String belongs to a recursive language
either accepts or rejects on Turing
machine.\\

\vspace*{0.2cm}
6. Find the decidable problem regarding the
DFA.\\

\vspace*{0.1cm}
\hspace*{0.5cm} a) The problem that a set of null strings is
accepted by a DFA M\\

\vspace*{0.1cm}
\hspace*{0.5cm} b) The problem that a string w is accepted
by a DFA M\\

\vspace*{0.1cm}
\hspace*{0.5cm} c) The problem that two DFA, M1 and M2,
satisfy the same language\\

\vspace*{0.1cm}
\hspace*{0.5cm} d) All of these\\


7. Which is true for reducibility?\\

\vspace*{0.1cm}
\hspace*{0.5cm} a) Converting one problem to another
problem.\\

\vspace*{0.1cm}
\hspace*{0.5cm} b) Converting one solved problem to
another unsolved problem.\\

\vspace*{0.1cm}
\hspace*{0.5cm} c) Converting one unsolved problem into
another solved problem to solve the
first problem.\\

\vspace*{0.1cm}
\hspace*{0.5cm} d) Converting one unsolved problem to
another unsolved problem.\\
\end{multicols}

\vspace*{0.3cm}

$Answers:$\\
1. d  \hspace*{0.5cm}  2. d   \hspace*{0.5cm}  3. a   \hspace*{0.5cm}  4. b   \hspace*{0.5cm}  5. c   \hspace*{0.5cm}  6. d   \hspace*{0.5cm}  7. c\\

\vspace*{0.3cm}
\begin{center}
\section{picture}
\includegraphics[width=12cm,height=0.7cm]{521.png}
\end{center}

\hspace*{-0.4cm}
1. Which of the following statements is false?\\

\vspace*{0.1cm}
\hspace*{0.5cm} a) The halting problem of a Turing machine is undecidable.\\

\vspace*{0.1cm}
\hspace*{0.5cm} b) Determining whether a context-free grammar is ambiguous is undecidable.\\

\vspace*{0.1cm}
\hspace*{0.5cm} c) Given two arbitrary context-free grammars $G _{1}$ and $G _{2}$, it is undecidable whether $L(G _{1}) =
L(G _{2})$.\\

\vspace*{0.1cm}
\hspace*{0.5cm} d) Given two regular grammars $G _{1}$ and $G _{2}$, it is undecidable whether $L(G _{1}) = L(G _{2})$.\\

\vspace*{0.2cm}

\hspace*{-0.4cm}
2. Which of the following is not decidable?\\

\vspace*{0.1cm}
\hspace*{0.5cm} a) Given a Turing machine M, a string s, and an integer k, M accepts s within k steps\\

\vspace*{0.1cm}
\hspace*{0.5cm} b) The equivalence of two given Turing machines\\

\vspace*{0.1cm}

\hspace*{0.5cm} c) The language accepted by a given finite state machine is not empty.\\

\vspace*{0.1cm}
\hspace*{0.5cm} d) The language generated by a context-free grammar is non-empty.\\

\vspace*{0.2cm}
\hspace*{-0.4cm}
3. Consider the following decision problems:\\

\vspace*{0.1cm}
\hspace*{0.5cm} $P _{1}$: Does a given finite state machine accept a given string\\

\vspace*{0.1cm}
\hspace*{0.5cm} $P _{2}$: Does a given context-free grammar generate an infinite number of strings.\\

\vspace*{0.1cm}
\hspace*{0.5cm} Which of the following statements is true?\\

\vspace*{0.1cm}
\hspace*{0.5cm} a) Both $P _{1}$ and $P _{2}$ are decidable\\
\hspace*{0.5cm} b) Neither $P _{1}$ nor $P _{2}$ are decidable\\
\hspace*{0.5cm} c) Only $P _{1}$ is decidable\\
\hspace*{0.5cm} d) Only $P _{2}$ is decidable\\

\vspace*{0.2cm}
\hspace*{-0.4cm}
4. Consider the following problem X.\\
\hspace*{0.5cm} Given a Turing machine $M$ over the input alphabet $\Sigma$, any state q of M and a word $w \in \Sigma ^{*}$, does
the computation of M on w visit the state $q$?\\

\newpage
\begin{flushleft}
    \textbf{522}\hspace*{0.1cm} \textbf{$|$} \hspace*{0.1cm} \texttt{Introduction to Automata Theory, Formal Languages and Computation}
  \end{flushleft}
  \vspace*{0.5cm}

\hspace*{0.5cm} a) X is decidable\\
\hspace*{0.5cm} b) X is undecidable but partially decidable\\
\hspace*{0.5cm} c) X is undecidable and not even partially decidable\\
\hspace*{0.5cm} d) X is not a decision problem\\
\vspace*{0.2cm}

\hspace*{-0.4cm}
5. Which of the following is true?\\
\vspace*{0.1cm}

\hspace*{0.5cm} a) The complement of a recursive language is recursive\\
\hspace*{0.5cm} b) The complement of a recursively enumerable language is recursively enumerable\\
\hspace*{0.5cm} c) The complement of a recursive language is either recursive or recursively enumerable\\
\hspace*{0.5cm} d) The complement of a context-free language is context free\\
\vspace*{0.2cm}

\hspace*{-0.4cm}
6. $L _{1}$ is a recursively enumerable language over $\Sigma$. An algorithm A effectively enumerates its words
as $w _{1}, w _{2}, w _{3}, . . .$ Define another language $L _{2}$ over $\Sigma \cup \{\#\}$ as $\{w _{i} \# w _{j}: w _{j} \in L _{1}, i < j\}$. Here, $\#$ is a new symbol.\\

\vspace*{0.1cm}
\hspace*{0.5cm} Consider the following assertions.\\
\vspace*{0.1cm}

\hspace*{0.5cm} $S _{1}: L _{1}$ is recursive implies that $L _{2}$ is recursive\\
\vspace*{0.1cm}

\hspace*{0.5cm} $S _{2}: L _{2}$ is recursive implies that $L _{1}$ is recursive
Which of the following statements is true?\\
\vspace*{0.1cm}

\hspace*{0.5cm} a) Both $S _{1}$ and $S _{2}$ are true\\
\hspace*{0.5cm} b) $S _{1}$ is true but $S _{2}$ is not necessarily true\\
\hspace*{0.5cm} c) $S _{2}$ is true but $S _{1}$ is not necessarily true\\
\hspace*{0.5cm} d) Neither is necessarily true\\
\vspace*{0.2cm}

\hspace*{-0.4cm}
7. Consider three decision problems $P _{1},P _{2}$, and $P _{3}$. It is known that $P _{1}$ is decidable and $P _{2}$ is undecidable.
Which one of the following is true?\\
\hspace*{0.5cm} a) $P _{3}$ is decidable if $P _{1}$ is reducible to $P _{3}$ \\
\hspace*{0.5cm} b) $P _{3}$ is undecidable if $P _{3}$ is reducible to $P _{2}$ \\
\hspace*{0.5cm} c) $P _{3}$ is undecidable if $P _{2}$ is reducible to $P _{3}$ \\
\hspace*{0.5cm} d) $P _{3}$ is decidable if $P _{3}$ is reducible to $P _{2}$’s complement\\
\vspace*{0.2cm}


\hspace*{-0.4cm}
8. Let L1 be a recursive language, and let $L _{2}$ be a recursively enumerable but not a recursive language.
Which one of the following is true?\\

\vspace*{0.1cm}
\hspace*{0.5cm} a) $-L _{1}$ is recursive and$-L _{2}$ is recursively enumerable.\\
\hspace*{0.5cm} b) $-L _{1}$ is recursive and$-L _{2}$ is not recursively enumerable.\\
\hspace*{0.5cm} c) $-L _{1}$ and $-L _{2}$ are recursively enumerable.\\
\hspace*{0.5cm} d) $-L _{1}$ is recursively enumerable and $-L _{2}$ is recursive\\
\vspace*{0.2cm}

\hspace*{-0.4cm}
9. For $S \in (0 + 1)^{*}$, let $d(s)$ denote the decimal value of s [e.g., $d(101) = 5]$. Let $L = \{s \in (0 + 1)^{*} |
d(s)$ mod $5 = 2$ and $d(s)$ mod $7 \neq 4\}$\\

\vspace*{0.1cm}
\hspace*{0.5cm} Which one of the following statements is true?\\
\hspace*{0.5cm} a) L is recursively enumerable, but not recursive\\
\hspace*{0.5cm} b) L is recursive, but not context free\\
\hspace*{0.5cm} c) L is context free, but not regular\\
\hspace*{0.5cm} d) L is regular\\

\newpage
\begin{flushright}
 \texttt{Computability and Undecidability} \hspace*{0.1cm}\textbf{$|$} \hspace*{0.1cm} \textbf{523}\hspace*{0.1cm}
\end{flushright}
\vspace*{0.5cm}

\hspace*{-0.4cm}
10. Let L1 be a regular language, L2 be a deterministic context-free language, and L3 a recursively
enumerable, but not recursive language. Which one of the following statements is false?\\

\vspace*{0.1cm}
\hspace*{0.5cm} a) $L _{1} \cap L _{2}$ is a deterministic CFL\\
\hspace*{0.5cm} b) $L _{3} \cap L _{1}$ is recursive\\
\hspace*{0.5cm} c) $L _{1} \cup L _{2}$ is context free\\
\hspace*{0.5cm} d) $L _{1} \cap L _{2} \cap L _{3}$ is recursively enumerable\\
\vspace*{0.2cm}

\hspace*{-0.4cm}
11. Which of the following problems is undecidable?\\
\vspace*{0.1cm}

\hspace*{0.5cm} a) Membership problem for CFGs. \hspace*{1cm} b) Ambiguity problem for CFGs.\\
\hspace*{0.5cm} c) Finiteness problem for FSAs  \hspace*{1.4cm} d) Equivalence problem for FSAs.\\
\vspace*{0.2cm}

\hspace*{-0.4cm}
12. The language $L = \{0 ^{i} 21 ^{i} | i \geq 0\}$ over alphabet $\{0, 1, 2\}$ is\\
\vspace*{0.1cm}

\hspace*{0.5cm} a) not recursive  \hspace*{1.8cm} b) recursive and is a deterministic CFL.\\
\hspace*{0.5cm} c) a regular language. \hspace*{1cm}  d) not a deterministic CFL but a CFL.\\
\vspace*{0.2cm}


\hspace*{-0.4cm}
13. Which of the following are decidable?\\
\hspace*{0.5cm} i) Whether the intersection of two regular languages is infinite\\
\hspace*{0.5cm} ii) Whether a given context-free language is regular\\
\hspace*{0.5cm} iii) Whether two pushdown automata accept the same language\\
\hspace*{0.5cm} iv) Whether a given grammar is context free\\
\hspace*{0.5cm} a) (i) and (ii)   \hspace*{0.5cm}   b) (i) and (iv)   \hspace*{0.5cm}   c) (ii) and (iii)  \hspace*{0.5cm}   d) (ii) and (iv)\\
\vspace*{0.2cm}

\hspace*{-0.4cm}
14. If L and$-\overline{L}$ are recursively enumerable, then $L$ is\\
\hspace*{0.5cm} a) Regular    \hspace*{0.5cm}  b) Context free    \hspace*{0.5cm}  c) Context sensitive    \hspace*{0.5cm}  d) Recursive\\
\vspace*{0.2cm}

\hspace*{-0.4cm}
15. Which of the following statements is false?\\
\hspace*{0.5cm} a) Every NFA can be converted to an equivalent DFA\\
\hspace*{0.5cm} b) Every non-deterministic Turing machine can be converted to an equivalent deterministic
Turing machine.\\
\hspace*{0.5cm} c) Every regular language is also a context-free language\\
\hspace*{0.5cm} d) Every subset of a recursively enumerable set is recursive.\\
\vspace*{0.2cm}


16. Let $L _{1}$ be a recursive language. Let $L _{2}$ and $L _{3}$ be the languages that are recursively enumerable but
not recursive. Which of the following statements is not necessarily true?\\
\hspace*{0.5cm} a) $L _{2} - L _{1}$ is recursively enumerable\\
\hspace*{0.5cm} b) $L _{1} - L _{3}$ is recursively enumerable\\
\hspace*{0.5cm} c) $L _{2} \cap L _{1}$ is recursively enumerable\\
\hspace*{0.5cm} d) $L _{2} \cup L _{1}$ is recursively enumerable\\
\vspace*{0.2cm}

\hspace*{-0.4cm}
17. Let $L = L _{1} \cap L _{2}$, where $L _{1}$ and $L _{2}$ are languages as defined in the following:\\
\vspace*{0.1cm}

\begin{center}
  $L ^{1} = \{ambmcanbn, m, n \geq 0), L ^{2} = \}a ^{i} b ^{i} c ^{k}, i, j, k \geq 0)$\\
\end{center}

Then, L is\\

\vspace*{0.1cm}
\hspace*{0.5cm} a) not recursive  \hspace*{2.5cm}  b) regular\\
\hspace*{0.5cm} c) context-free but not regular   \hspace*{0.5cm} d) recursively enumerable but not context free\\

\newpage
\begin{flushleft}
    \textbf{524}\hspace*{0.1cm} \textbf{$|$} \hspace*{0.1cm} \texttt{Introduction to Automata Theory, Formal Languages and Computation}
  \end{flushleft}
  \vspace*{0.5cm}

\hspace*{-0.4cm}
18. Which of the following problems are decidable?\\
\vspace*{0.1cm}

\hspace*{0.5cm} i) Does a given program ever produce an output?\\
\hspace*{0.5cm} ii) If L is a context-free language, then, is $-L$ also context free?\\
\hspace*{0.5cm} iii) If L is a regular language, then is $-L$ also regular?\\
\hspace*{0.5cm} iv) If L is a recursive language, then is $-L$ also recursive?\\
\vspace*{0.1cm}

\hspace*{0.5cm} a) (i), (ii), (iii), (iv)   \vspace*{0.1cm}  b) (i), (ii)  \vspace*{0.1cm} c) (ii), (iii), (iv)   \vspace*{0.1cm}  d) (iii), (iv)\\
\vspace*{0.3cm}

\hspace*{-0.5cm}
Answers:\\
\hspace*{0.2cm} 1. d   \hspace*{0.5cm}  2. b  \hspace*{0.5cm}  3. a   \hspace*{0.5cm}  4. b   \hspace*{0.5cm}  5. a   \hspace*{0.5cm}  6. b  \hspace*{0.5cm}  7. b  \hspace*{0.5cm} 8. b   \hspace*{0.5cm} 9. b
\hspace*{0.5cm} 10. b  \hspace*{0.5cm} 11. b  \hspace*{0.5cm} 12. b   \hspace*{0.5cm} 13. b   \hspace*{0.5cm} 14. d   \hspace*{0.5cm}  15. d   \hspace*{0.5cm} 16. b  \hspace*{0.5cm} 17. a  \hspace*{0.5cm} 18. d\\

\vspace*{0.4cm}

\hspace*{-0.5cm}
\textbf{Hints:}

\vspace*{0.1cm}
1. From a regular grammar, FA can be designed. It can be tested whether two FA are equivalent
or not.\\

\vspace*{0.2cm}
3. A given CFG is infinite if there is at least one cycle in the directed graph generated from the
production rules of the given CFG in CNF.\\

\vspace*{0.2cm}
4. The problem is undecidable. But if the state is the beginning state, it must be traversed. Thus, it is
partially decidable.\\

\vspace*{0.2cm}
8. The complement of a recursively enumerable language is not recursively enumerable.\\

\vspace*{0.2cm}
9. $L = 5n + 2$ but $L \neq 7n + 4$. Hence, we can design a machine which halts and accepts if $L = 5n + 2$
and halts and rejects if $L \neq 7n + 4$. So, it is decidable.\\

\vspace*{0.2cm}
10. $L _{1}$ and $L _{2}$ are recursive. The intersection of two recursive languages is recursive. But the intersection
of the recursive and the recursively enumerable languages are recursively enumerable and
not recursive.\\

\vspace*{0.2cm}

12. A DPDA can be designed which PUSH X for each appearance of ‘0’. No stack operation for
traversing 2 and POP X for ‘1’. A Turing machine can be designed for it where for each ‘0’ it
traverses the rightmost ‘1’ by replacing them by X and Y, respectively. If after X, 2 appears and
after 2, Y appears, then it halts. Thus, it is recursive.\\
\vspace*{0.2cm}

13. The intersection of two regular language is regular, i.e., CFL. Using $Q3$, we can find infiniteness.\\
\vspace*{0.2cm}

16. The size of the $L _{1}$ set is less than the size of the $L _{3}$ set.\\
\vspace*{0.2cm}

17. Two languages are CFL. The intersection of the two CFL is not a CFL, so not regular. (b and c are
false). The answer will be ‘a’ or ‘d’.\\
\vspace*{0.1cm}

\begin{center}
\section{picture}
\includegraphics[width=12cm,height=0.5cm]{524.png}
\end{center}

1. Prove that any decision cannot be taken for $A \cup B ^{C}$, if A is recursive and B is recursively enumerable.\\
\vspace*{0.2cm}

2. Prove that $A \cup B ^{C}$ is recursively enumerable if A is recursive and B is recursively enumerable.\\

\end{document} 